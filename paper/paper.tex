
%% bare_jrnl_compsoc.tex
%% V1.4b
%% 2015/08/26
%% by Michael Shell
%% See:
%% http://www.michaelshell.org/
%% for current contact information.
%%
%% This is a skeleton file demonstrating the use of IEEEtran.cls
%% (requires IEEEtran.cls version 1.8b or later) with an IEEE
%% Computer Society journal paper.
%%
%% Support sites:
%% http://www.michaelshell.org/tex/ieeetran/
%% http://www.ctan.org/pkg/ieeetran
%% and
%% http://www.ieee.org/

%%*************************************************************************
%% Legal Notice:
%% This code is offered as-is without any warranty either expressed or
%% implied; without even the implied warranty of MERCHANTABILITY or
%% FITNESS FOR A PARTICULAR PURPOSE! 
%% User assumes all risk.
%% In no event shall the IEEE or any contributor to this code be liable for
%% any damages or losses, including, but not limited to, incidental,
%% consequential, or any other damages, resulting from the use or misuse
%% of any information contained here.
%%
%% All comments are the opinions of their respective authors and are not
%% necessarily endorsed by the IEEE.
%%
%% This work is distributed under the LaTeX Project Public License (LPPL)
%% ( http://www.latex-project.org/ ) version 1.3, and may be freely used,
%% distributed and modified. A copy of the LPPL, version 1.3, is included
%% in the base LaTeX documentation of all distributions of LaTeX released
%% 2003/12/01 or later.
%% Retain all contribution notices and credits.
%% ** Modified files should be clearly indicated as such, including  **
%% ** renaming them and changing author support contact information. **
%%*************************************************************************


% *** Authors should verify (and, if needed, correct) their LaTeX system  ***
% *** with the testflow diagnostic prior to trusting their LaTeX platform ***
% *** with production work. The IEEE's font choices and paper sizes can   ***
% *** trigger bugs that do not appear when using other class files.       ***                          ***
% The testflow support page is at:
% http://www.michaelshell.org/tex/testflow/


\documentclass[10pt,journal,compsoc]{IEEEtran}
%
% If IEEEtran.cls has not been installed into the LaTeX system files,
% manually specify the path to it like:
% \documentclass[10pt,journal,compsoc]{../sty/IEEEtran}





% Some very useful LaTeX packages include:
% (uncomment the ones you want to load)


% *** MISC UTILITY PACKAGES ***
%
%\usepackage{ifpdf}
% Heiko Oberdiek's ifpdf.sty is very useful if you need conditional
% compilation based on whether the output is pdf or dvi.
% usage:
% \ifpdf
%   % pdf code
% \else
%   % dvi code
% \fi
% The latest version of ifpdf.sty can be obtained from:
% http://www.ctan.org/pkg/ifpdf
% Also, note that IEEEtran.cls V1.7 and later provides a builtin
% \ifCLASSINFOpdf conditional that works the same way.
% When switching from latex to pdflatex and vice-versa, the compiler may
% have to be run twice to clear warning/error messages.
\usepackage[english]{babel}
\usepackage{blindtext}




% *** CITATION PACKAGES ***
%
\ifCLASSOPTIONcompsoc
  % IEEE Computer Society needs nocompress option
  % requires cite.sty v4.0 or later (November 2003)
  \usepackage[nocompress]{cite}
\else
  % normal IEEE
  \usepackage{cite}
\fi
% cite.sty was written by Donald Arseneau
% V1.6 and later of IEEEtran pre-defines the format of the cite.sty package
% \cite{} output to follow that of the IEEE. Loading the cite package will
% result in citation numbers being automatically sorted and properly
% "compressed/ranged". e.g., [1], [9], [2], [7], [5], [6] without using
% cite.sty will become [1], [2], [5]--[7], [9] using cite.sty. cite.sty's
% \cite will automatically add leading space, if needed. Use cite.sty's
% noadjust option (cite.sty V3.8 and later) if you want to turn this off
% such as if a citation ever needs to be enclosed in parenthesis.
% cite.sty is already installed on most LaTeX systems. Be sure and use
% version 5.0 (2009-03-20) and later if using hyperref.sty.
% The latest version can be obtained at:
% http://www.ctan.org/pkg/cite
% The documentation is contained in the cite.sty file itself.
%
% Note that some packages require special options to format as the Computer
% Society requires. In particular, Computer Society  papers do not use
% compressed citation ranges as is done in typical IEEE papers
% (e.g., [1]-[4]). Instead, they list every citation separately in order
% (e.g., [1], [2], [3], [4]). To get the latter we need to load the cite
% package with the nocompress option which is supported by cite.sty v4.0
% and later. Note also the use of a CLASSOPTION conditional provided by
% IEEEtran.cls V1.7 and later.





% *** GRAPHICS RELATED PACKAGES ***
%
\ifCLASSINFOpdf
  \usepackage[pdftex]{graphicx}
  % declare the path(s) where your graphic files are
  % \graphicspath{{../pdf/}{../jpeg/}}
  % and their extensions so you won't have to specify these with
  % every instance of \includegraphics
  \DeclareGraphicsExtensions{.pdf,.jpeg,.png}
\else
  % or other class option (dvipsone, dvipdf, if not using dvips). graphicx
  % will default to the driver specified in the system graphics.cfg if no
  % driver is specified.
  % \usepackage[dvips]{graphicx}
  % declare the path(s) where your graphic files are
  % \graphicspath{{../eps/}}
  % and their extensions so you won't have to specify these with
  % every instance of \includegraphics
  % \DeclareGraphicsExtensions{.eps}
\fi
% graphicx was written by David Carlisle and Sebastian Rahtz. It is
% required if you want graphics, photos, etc. graphicx.sty is already
% installed on most LaTeX systems. The latest version and documentation
% can be obtained at: 
% http://www.ctan.org/pkg/graphicx
% Another good source of documentation is "Using Imported Graphics in
% LaTeX2e" by Keith Reckdahl which can be found at:
% http://www.ctan.org/pkg/epslatex
%
% latex, and pdflatex in dvi mode, support graphics in encapsulated
% postscript (.eps) format. pdflatex in pdf mode supports graphics
% in .pdf, .jpeg, .png and .mps (metapost) formats. Users should ensure
% that all non-photo figures use a vector format (.eps, .pdf, .mps) and
% not a bitmapped formats (.jpeg, .png). The IEEE frowns on bitmapped formats
% which can result in "jaggedy"/blurry rendering of lines and letters as
% well as large increases in file sizes.
%
% You can find documentation about the pdfTeX application at:
% http://www.tug.org/applications/pdftex






% *** MATH PACKAGES ***
%
%\usepackage{amsmath}
% A popular package from the American Mathematical Society that provides
% many useful and powerful commands for dealing with mathematics.
%
% Note that the amsmath package sets \interdisplaylinepenalty to 10000
% thus preventing page breaks from occurring within multiline equations. Use:
%\interdisplaylinepenalty=2500
% after loading amsmath to restore such page breaks as IEEEtran.cls normally
% does. amsmath.sty is already installed on most LaTeX systems. The latest
% version and documentation can be obtained at:
% http://www.ctan.org/pkg/amsmath





% *** SPECIALIZED LIST PACKAGES ***
%
%\usepackage{algorithmic}
% algorithmic.sty was written by Peter Williams and Rogerio Brito.
% This package provides an algorithmic environment fo describing algorithms.
% You can use the algorithmic environment in-text or within a figure
% environment to provide for a floating algorithm. Do NOT use the algorithm
% floating environment provided by algorithm.sty (by the same authors) or
% algorithm2e.sty (by Christophe Fiorio) as the IEEE does not use dedicated
% algorithm float types and packages that provide these will not provide
% correct IEEE style captions. The latest version and documentation of
% algorithmic.sty can be obtained at:
% http://www.ctan.org/pkg/algorithms
% Also of interest may be the (relatively newer and more customizable)
% algorithmicx.sty package by Szasz Janos:
% http://www.ctan.org/pkg/algorithmicx




% *** ALIGNMENT PACKAGES ***
%
%\usepackage{array}
% Frank Mittelbach's and David Carlisle's array.sty patches and improves
% the standard LaTeX2e array and tabular environments to provide better
% appearance and additional user controls. As the default LaTeX2e table
% generation code is lacking to the point of almost being broken with
% respect to the quality of the end results, all users are strongly
% advised to use an enhanced (at the very least that provided by array.sty)
% set of table tools. array.sty is already installed on most systems. The
% latest version and documentation can be obtained at:
% http://www.ctan.org/pkg/array


% IEEEtran contains the IEEEeqnarray family of commands that can be used to
% generate multiline equations as well as matrices, tables, etc., of high
% quality.




% *** SUBFIGURE PACKAGES ***
\ifCLASSOPTIONcompsoc
 \usepackage[caption=false,font=footnotesize,labelfont=sf,textfont=sf]{subfig}
\else
 \usepackage[caption=false,font=footnotesize]{subfig}
\fi
% subfig.sty, written by Steven Douglas Cochran, is the modern replacement
% for subfigure.sty, the latter of which is no longer maintained and is
% incompatible with some LaTeX packages including fixltx2e. However,
% subfig.sty requires and automatically loads Axel Sommerfeldt's caption.sty
% which will override IEEEtran.cls' handling of captions and this will result
% in non-IEEE style figure/table captions. To prevent this problem, be sure
% and invoke subfig.sty's "caption=false" package option (available since
% subfig.sty version 1.3, 2005/06/28) as this is will preserve IEEEtran.cls
% handling of captions.
% Note that the Computer Society format requires a sans serif font rather
% than the serif font used in traditional IEEE formatting and thus the need
% to invoke different subfig.sty package options depending on whether
% compsoc mode has been enabled.
%
% The latest version and documentation of subfig.sty can be obtained at:
% http://www.ctan.org/pkg/subfig




% *** FLOAT PACKAGES ***
%
%\usepackage{fixltx2e}
% fixltx2e, the successor to the earlier fix2col.sty, was written by
% Frank Mittelbach and David Carlisle. This package corrects a few problems
% in the LaTeX2e kernel, the most notable of which is that in current
% LaTeX2e releases, the ordering of single and double column floats is not
% guaranteed to be preserved. Thus, an unpatched LaTeX2e can allow a
% single column figure to be placed prior to an earlier double column
% figure.
% Be aware that LaTeX2e kernels dated 2015 and later have fixltx2e.sty's
% corrections already built into the system in which case a warning will
% be issued if an attempt is made to load fixltx2e.sty as it is no longer
% needed.
% The latest version and documentation can be found at:
% http://www.ctan.org/pkg/fixltx2e


%\usepackage{stfloats}
% stfloats.sty was written by Sigitas Tolusis. This package gives LaTeX2e
% the ability to do double column floats at the bottom of the page as well
% as the top. (e.g., "\begin{figure*}[!b]" is not normally possible in
% LaTeX2e). It also provides a command:
%\fnbelowfloat
% to enable the placement of footnotes below bottom floats (the standard
% LaTeX2e kernel puts them above bottom floats). This is an invasive package
% which rewrites many portions of the LaTeX2e float routines. It may not work
% with other packages that modify the LaTeX2e float routines. The latest
% version and documentation can be obtained at:
% http://www.ctan.org/pkg/stfloats
% Do not use the stfloats baselinefloat ability as the IEEE does not allow
% \baselineskip to stretch. Authors submitting work to the IEEE should note
% that the IEEE rarely uses double column equations and that authors should try
% to avoid such use. Do not be tempted to use the cuted.sty or midfloat.sty
% packages (also by Sigitas Tolusis) as the IEEE does not format its papers in
% such ways.
% Do not attempt to use stfloats with fixltx2e as they are incompatible.
% Instead, use Morten Hogholm'a dblfloatfix which combines the features
% of both fixltx2e and stfloats:
%
% \usepackage{dblfloatfix}
% The latest version can be found at:
% http://www.ctan.org/pkg/dblfloatfix




%\ifCLASSOPTIONcaptionsoff
%  \usepackage[nomarkers]{endfloat}
% \let\MYoriglatexcaption\caption
% \renewcommand{\caption}[2][\relax]{\MYoriglatexcaption[#2]{#2}}
%\fi
% endfloat.sty was written by James Darrell McCauley, Jeff Goldberg and 
% Axel Sommerfeldt. This package may be useful when used in conjunction with 
% IEEEtran.cls'  captionsoff option. Some IEEE journals/societies require that
% submissions have lists of figures/tables at the end of the paper and that
% figures/tables without any captions are placed on a page by themselves at
% the end of the document. If needed, the draftcls IEEEtran class option or
% \CLASSINPUTbaselinestretch interface can be used to increase the line
% spacing as well. Be sure and use the nomarkers option of endfloat to
% prevent endfloat from "marking" where the figures would have been placed
% in the text. The two hack lines of code above are a slight modification of
% that suggested by in the endfloat docs (section 8.4.1) to ensure that
% the full captions always appear in the list of figures/tables - even if
% the user used the short optional argument of \caption[]{}.
% IEEE papers do not typically make use of \caption[]'s optional argument,
% so this should not be an issue. A similar trick can be used to disable
% captions of packages such as subfig.sty that lack options to turn off
% the subcaptions:
% For subfig.sty:
% \let\MYorigsubfloat\subfloat
% \renewcommand{\subfloat}[2][\relax]{\MYorigsubfloat[]{#2}}
% However, the above trick will not work if both optional arguments of
% the \subfloat command are used. Furthermore, there needs to be a
% description of each subfigure *somewhere* and endfloat does not add
% subfigure captions to its list of figures. Thus, the best approach is to
% avoid the use of subfigure captions (many IEEE journals avoid them anyway)
% and instead reference/explain all the subfigures within the main caption.
% The latest version of endfloat.sty and its documentation can obtained at:
% http://www.ctan.org/pkg/endfloat
%
% The IEEEtran \ifCLASSOPTIONcaptionsoff conditional can also be used
% later in the document, say, to conditionally put the References on a 
% page by themselves.




% *** PDF, URL AND HYPERLINK PACKAGES ***
%
\usepackage{url}
% url.sty was written by Donald Arseneau. It provides better support for
% handling and breaking URLs. url.sty is already installed on most LaTeX
% systems. The latest version and documentation can be obtained at:
% http://www.ctan.org/pkg/url
% Basically, \url{my_url_here}.





% *** Do not adjust lengths that control margins, column widths, etc. ***
% *** Do not use packages that alter fonts (such as pslatex).         ***
% There should be no need to do such things with IEEEtran.cls V1.6 and later.
% (Unless specifically asked to do so by the journal or conference you plan
% to submit to, of course. )


% correct bad hyphenation here
\hyphenation{op-tical net-works semi-conduc-tor}

\usepackage{xcolor}

\usepackage{color}

\usepackage[pdftex]{hyperref}
\newcommand{\figref}[1]{\hyperref[#1]{Figure~\ref*{#1}}}
\newcommand{\secref}[1]{\hyperref[#1]{Section~\ref*{#1}}}
\newcommand{\tabref}[1]{\hyperref[#1]{Table~\ref*{#1}}}
\newcommand{\ea}{{et~al.}\xspace}

\newcommand{\note}[2]{\textcolor{#1}{[#2]}}
\newcommand{\todo}[1]{\note{red}{TODO: #1}}


\begin{document}
%
% paper title
% Titles are generally capitalized except for words such as a, an, and, as,
% at, but, by, for, in, nor, of, on, or, the, to and up, which are usually
% not capitalized unless they are the first or last word of the title.
% Linebreaks \\ can be used within to get better formatting as desired.
% Do not put math or special symbols in the title.
\title{Examining Whether Approximate Visualizations Impact Perception of Corrected Visualizations}
%
%
% author names and IEEE memberships
% note positions of commas and nonbreaking spaces ( ~ ) LaTeX will not break
% a structure at a ~ so this keeps an author's name from being broken across
% two lines.
% use \thanks{} to gain access to the first footnote area
% a separate \thanks must be used for each paragraph as LaTeX2e's \thanks
% was not built to handle multiple paragraphs
%
%
%\IEEEcompsocitemizethanks is a special \thanks that produces the bulleted
% lists the Computer Society journals use for "first footnote" author
% affiliations. Use \IEEEcompsocthanksitem which works much like \item
% for each affiliation group. When not in compsoc mode,
% \IEEEcompsocitemizethanks becomes like \thanks and
% \IEEEcompsocthanksitem becomes a line break with idention. This
% facilitates dual compilation, although admittedly the differences in the
% desired content of \author between the different types of papers makes a
% one-size-fits-all approach a daunting prospect. For instance, compsoc 
% journal papers have the author affiliations above the "Manuscript
% received ..."  text while in non-compsoc journals this is reversed. Sigh.

\author{Dominik~Moritz
        and~Daniel~A.~Epstein
\IEEEcompsocitemizethanks{\IEEEcompsocthanksitem Dominik Moritz and Daniel Epstein are with the University of Washington.}% <-this % stops an unwanted space
\thanks{Manuscript received November 14, 2016.}}

% note the % following the last \IEEEmembership and also \thanks - 
% these prevent an unwanted space from occurring between the last author name
% and the end of the author line. i.e., if you had this:
% 
% \author{....lastname \thanks{...} \thanks{...} }
%                     ^------------^------------^----Do not want these spaces!
%
% a space would be appended to the last name and could cause every name on that
% line to be shifted left slightly. This is one of those "LaTeX things". For
% instance, "\textbf{A} \textbf{B}" will typeset as "A B" not "AB". To get
% "AB" then you have to do: "\textbf{A}\textbf{B}"
% \thanks is no different in this regard, so shield the last } of each \thanks
% that ends a line with a % and do not let a space in before the next \thanks.
% Spaces after \IEEEmembership other than the last one are OK (and needed) as
% you are supposed to have spaces between the names. For what it is worth,
% this is a minor point as most people would not even notice if the said evil
% space somehow managed to creep in.



% The paper headers
\markboth{Class paper for CSE599E1, User studies in software engineering}%
{Moritz and Epstein, CSE599E1}
% The only time the second header will appear is for the odd numbered pages
% after the title page when using the twoside option.
% 
% *** Note that you probably will NOT want to include the author's ***
% *** name in the headers of peer review papers.                   ***
% You can use \ifCLASSOPTIONpeerreview for conditional compilation here if
% you desire.



% The publisher's ID mark at the bottom of the page is less important with
% Computer Society journal papers as those publications place the marks
% outside of the main text columns and, therefore, unlike regular IEEE
% journals, the available text space is not reduced by their presence.
% If you want to put a publisher's ID mark on the page you can do it like
% this:
%\IEEEpubid{0000--0000/00\$00.00~\copyright~2015 IEEE}
% or like this to get the Computer Society new two part style.
%\IEEEpubid{\makebox[\columnwidth]{\hfill 0000--0000/00/\$00.00~\copyright~2015 IEEE}%
%\hspace{\columnsep}\makebox[\columnwidth]{Published by the IEEE Computer Society\hfill}}
% Remember, if you use this you must call \IEEEpubidadjcol in the second
% column for its text to clear the IEEEpubid mark (Computer Society jorunal
% papers don't need this extra clearance.)



% use for special paper notices
%\IEEEspecialpapernotice{(Invited Paper)}



% for Computer Society papers, we must declare the abstract and index terms
% PRIOR to the title within the \IEEEtitleabstractindextext IEEEtran
% command as these need to go into the title area created by \maketitle.
% As a general rule, do not put math, special symbols or citations
% in the abstract or keywords.
\IEEEtitleabstractindextext{%
\begin{abstract}
Data analysts often have to work with an approximate model before more precise results are available.
However, after they have seen the accurate results, are they still biased by the approximation?
In this paper we present a controlled experiment with 34 participants on Mechanical Turk to investigate whether exposure to approximate results biases user's recollection of the precise data.
In our study, we aimed to simulate an exploratory data analysis session with an approximate query processing (AQP) system, a common class of systems used to accelerate data exploration.
Our goal was to understand whether analysts are biased by approximate data and whether highlighting the difference reduces this bias.
We found that analyst's recollection of precise data is biased but have no conclusive results about whether highlighting differences reduces the bias.
The results from this paper are relevant for AQP systems and count extend to other applications in journalism and science.
\end{abstract}

% Note that keywords are not normally used for peerreview papers.
\begin{IEEEkeywords}
Approximate visualizations, perception
\end{IEEEkeywords}}


% make the title area
\maketitle


% To allow for easy dual compilation without having to reenter the
% abstract/keywords data, the \IEEEtitleabstractindextext text will
% not be used in maketitle, but will appear (i.e., to be "transported")
% here as \IEEEdisplaynontitleabstractindextext when the compsoc 
% or transmag modes are not selected <OR> if conference mode is selected 
% - because all conference papers position the abstract like regular
% papers do.
\IEEEdisplaynontitleabstractindextext
% \IEEEdisplaynontitleabstractindextext has no effect when using
% compsoc or transmag under a non-conference mode.



% For peer review papers, you can put extra information on the cover
% page as needed:
% \ifCLASSOPTIONpeerreview
% \begin{center} \bfseries EDICS Category: 3-BBND \end{center}
% \fi
%
% For peerreview papers, this IEEEtran command inserts a page break and
% creates the second title. It will be ignored for other modes.
\IEEEpeerreviewmaketitle



\IEEEraisesectionheading{\section{Introduction}\label{sec:introduction}}
% Computer Society journal (but not conference!) papers do something unusual
% with the very first section heading (almost always called "Introduction").
% They place it ABOVE the main text! IEEEtran.cls does not automatically do
% this for you, but you can achieve this effect with the provided
% \IEEEraisesectionheading{} command. Note the need to keep any \label that
% is to refer to the section immediately after \section in the above as
% \IEEEraisesectionheading puts \section within a raised box.




% The very first letter is a 2 line initial drop letter followed
% by the rest of the first word in caps (small caps for compsoc).
% 
% form to use if the first word consists of a single letter:
% \IEEEPARstart{A}{demo} file is ....
% 
% form to use if you need the single drop letter followed by
% normal text (unknown if ever used by the IEEE):
% \IEEEPARstart{A}{}demo file is ....
% 
% Some journals put the first two words in caps:
% \IEEEPARstart{T}{his demo} file is ....
% 
% Here we have the typical use of a "T" for an initial drop letter
% and "HIS" in caps to complete the first word.
\IEEEPARstart{A}{s} data science grows in interest and importance, data analysts want to derive insights from increasingly large datasets.
Depending on the dataset, it could take a system seconds or minutes to construct accurate summary visualizations of the data.
The time to generate these visualizations drastically impedes the process of exploratory data visualization \cite{liu2014effects}, where analysts often consider many visualizations in sequence.
A common technique to enable exploratory analysis of large datasets is to sample the data for immediate consumption \cite{moritz2017pangloss}.
Precise results considering the entire dataset are either updated progressively \cite{fisher2012trust} or asynchronously \cite{moritz2017pangloss, kay2016ish}.

By their nature, the immediate visualizations are not precise.
Errors can affect the observations analysts make and thus the conclusions they derive from the data.
This work considers the perceptual question of whether temporary exposure to imprecise visualizations in exploratory data analysis influences people's recognition of the data.
We further explore whether the design of the precise visualization can reduce the impact of the temporary exposure to imprecise data.

Specifically, this paper seeks to answer the following questions:
\begin{enumerate}
\item Does seeing an imprecise visualization generated from sampled data prior to a precise visualization impact people's recollection of the data viewed?
\item Is this impact mitigated by highlighting the change from the imprecise visualization in the precise visualization?
\end{enumerate}

To answer these questions, we conduct a controlled experiment simulating an exploratory data analysis process with a dataset of flights in the United States.
Participants first viewed two approximate visualizations generated by sampling the flight data, then viewed the same visualizations updated with the precise data.
Our experiment evaluated varying levels of error and different methods for highlighting the change in the precise visualizations.

Through an 38-person experiment on Amazon Mechanical Turk, we found that exposure to an incorrect visualization biases the recollection of data from the precise visualization.
However, we did not see enough evidence to support the hypothesis that highlighting the differences reduces this bias.

\section{Related Work}\label{sec:related_work}

This paper builds from literature on exploratory data analysis visualizing uncertainty, and studies of human perception of visualizations.

\subsection{Exploratory data analysis for large datasets}

In exploratory data analysis (EDA)~\cite{tukey1977exploratory} analysts example multi-dimensional data by looking at the distributions and correlations of fields.
This process is iterative and an analyst might look at dozens of graphs as they get to know the data~\cite{card1991information}. Some of these charts contain observations that allows analysts to move on to the next step of their analysis ~\cite{yi2008understanding}.
For instance, when examining a dataset of flights, an observation might be ``Most flights are out of New York State.''

A Visualization system must be fast enough to enable fast iterations; with delays of more than 500ms analysts become less effective ~\cite{liu2014effects} or even lose their flow of thought if delays are more than a second~\cite{nielsen1993response}.

To enable fast response times over large data, visualization system use sampling and approximation to reduce the amount of data that has to be evaluated to compute the results of a query~\cite{agarwal2013blinkdb,ding2016sample+,kamat2014distributed}.

The HCI community found that approximate visualizations can be used in exploratory data analysis~\cite{fisher2012trust, moritz2017pangloss}.
However, there is no prior work on how exposure to approximate visualizations biases the observations and insights analysts make.

\subsection{Visualizing Uncertainty and Human Perception of Visualizations}

There are a number of techniques for communicating uncertainty using visualizations~\cite{kay2016ish, olston2002visualizing}.
However, often users often struggle to correctly interpret the uncertainty or even draw incorrect conclusions~\cite{joslyn2013decisions}.
Moreover, in many models in the real world no statistical model is available to estimate the uncertainty so that the approximate visualizations in our study do not show uncertainties.

Borkin et.al. \cite{borkin2013makes} and others have studied the memorability of data visualizations.
Prior work distinguishes memorability of the visualization itself and the underlying data.
In this paper, we only focus on what data users remember because the mental models users build of the data are what influences their decisions on what to look at next.
To the best of our knowledge there are no studies on the biases that exposure to imprecise visualizations introduces.

\section{Experiment Description}\label{sec:experimental_setup}

\subsection{Research Questions}

The questions we want to answer two questions about participants recollection of the data form a precise data when they see an approximate, incorrect visualization first. 

\begin{enumerate}
  \item[\textbf{RQ1}]: Do participants bias their recollection towards the data from the approximate visualization?
  \item[\textbf{RQ2}]: Is this bias reduced by showing participants the approximate and precise data or a visualization of the difference as opposed to only the precise data?
\end{enumerate}

We designed our study to answer these two questions in a controlled experiment that mimics a real data analysis session.
RQ2 of course only makes sense if the answer to RQ1 is yes but we have reason to believe it is.

\subsection{Experimental Design}

\begin{figure*}[!ht]
\centering
\subfloat[Instructions for the study in general.]{\includegraphics[width=0.18\textwidth]{1.png}%
\label{fig:one}}
\hfil
\subfloat[First approximate visualization (here airlines).]{\includegraphics[width=0.18\textwidth]{2.png}%
\label{fig:two}}
\hfil
\subfloat[Questions about first approximate visualization.]{\includegraphics[width=0.18\textwidth]{3.png}%
\label{fig:three}}
\hfil
\subfloat[Second approximate visualization (here states).]{\includegraphics[width=0.18\textwidth]{4.png}%
\label{fig:four}}
\hfil
\subfloat[Questions about second approximate visualization.]{\includegraphics[width=0.18\textwidth]{5.png}%
\label{fig:five}}
\hfil
\subfloat[Instructions for the precise visualizations.]{\includegraphics[width=0.18\textwidth]{6.png}%
\label{fig:six}}
\hfil
\subfloat[First precise visualization (here airlines and difference visualization).]{\includegraphics[width=0.18\textwidth]{7.png}%
\label{fig:seven}}
\hfil
\subfloat[Second precise visualization (here states and difference visualization).]{\includegraphics[width=0.18\textwidth]{8.png}%
\label{fig:eight}}
\hfil
\subfloat[Questions about both precise visualizations.]{\includegraphics[width=0.18\textwidth]{9.png}%
\label{fig:nine}}
\hfil
\subfloat[Questions about the demographics.]{\includegraphics[width=0.18\textwidth]{10.png}%
\label{fig:ten}}

\caption{A possible experimental flow with two visualizations about airlines and states with a precise visualization that shows the difference between the approximate and precise data.}
\label{fig:example}
\end{figure*}

We sought to create an experiment which mimicked the experience of Exploratory Data Analysis with an approximate query system while still maintaining control over participant experiences.
In approximate query systems, users often internalize approximate visualizations, using the gained knowledge to inform the next visualization to create.
Systems, such as Pangloss \cite{moritz2017pangloss}, can later generate visualizations of the precise results.
However, the user may have moved on to investigate a new visualization before the precise result is computed.

The study is online at \url{https://domoritz.github.io/bias-study/index.html}.

\subsubsection{Study Flow}

Our study flow (\figref{fig:example}) aimed to mimic the experience of a data scientist analyzing a large dataset about flights in an approximate query processing system.
Participants viewed two approximate visualizations followed by two precise visualizations and then asked questions about the data in them.
Before we showed the first visualization, we instructed each participant to focus on a particular airline and state.
This mimics a data analysis session where analysts form hypotheses that affect what they focus on rather than trying to look at every possible aspect (e.g. airlines, states) with equal attention.
The study enforced participants view the visualizations for at least 10 seconds before advancing. %TODO: why do we pick 10 seconds?
To ensure that participants had considered the approximate data, we asked a few questions after viewing each approximate visualization informed by Brehmer and Munzner's work \cite{brehmer2013multi}.
Answering these questions simulated how a data analyst would consider and reflect on their data.
The questions also provided us a baseline level of comprehension of visualizations of our participants.
After viewing both precise visualizations, participants answered a larger set of questions similar to the ones we asked about the visualization of the approximate data.
Participants were asked to rate their confidence in their answer to each question in both sets on a sliding scale from ``I don't remember'' to ``I'm certain''.
Participants were informed that the first two visualizations showed approximations of their data, allowing us to mimic the experience of looking at an approximate query result.

\subsubsection{Visualizations in the Study}\label{sec:vizes}

The visualizations participants saw were based on the same dataset about flights in the United States \footnote{\url{http://datahub.io/dataset/open-flights}}.
All visualizations were bar graphs, which are common in approximate query systems \cite{moritz2017pangloss}.
The y-axis varied between the two visualizations, showing either the state the flight left from or the airline the flight was operated by.
The length of the bar encodes the number of flights operated by a particular airline or the number of flights out of different states.
The bars were sorted in descending order so that the state or airline with the most flights is at the top.
We chose horizontal bar charts so that long labels are easier to read.
To visually separate visualizations of approximate and precise data, we colored the bars approximate visualizations orange (\figref{fig:approx}) and precise visualizations blue (\figref{fig:precise}). 

We used one month of flight data to generate the visualizations.
The precise visualization of number of flights per state uses data where the airline was ``JetBlue'' and the visualization of the flight per airline uses data from ``New York State''.
The approximate visualizations use a random subset of the data for the precise visualizations with replacement.
We then computed the approximate values by dividing the count values by the fraction of rows used in the sample.

In the visualizations of approximate data, the values can be different because the data is based on a sample.
Groups can be missing entirely if there was no example of a particular state or airline in the random sample (e.g. \figref{fig:approx}).
Moreover, the order of bars can be incorrect because we always sort by the values.
The questions we ask participants of our study are designed to fit these errors that can occur.

\begin{figure*}[!ht]
\centering
\subfloat[Approximate visualization based on 0.2\% of the data. Note that ``New Jersey'' is missing and that ``Florida'' is the state with the most outgoing flights.]{\includegraphics[width=0.3\textwidth]{states5009.png}%
\label{fig:approx}}
\hfil
\subfloat[Visualization of precise data.]{\includegraphics[width=0.3\textwidth]{states10.png}%
\label{fig:precise}}
\hfil
\subfloat[Precise with difference visualization. The orange ticks show the user the approximate data they have previously seen.]{\includegraphics[width=0.3\textwidth]{states5009diff.png}%
\label{fig:diff}}
\caption{Visualizations of approximate and precise data of the number of flight per state. (c) highlights the difference to the approximate data.}
\label{fig:types}
\end{figure*}

We designed three different visualizations of the precise data to understand whether they affect the bias (RQ2).
The baseline condition was to show only the precise data (\figref{fig:precise}).
We could also show participants both the approximate (\figref{fig:approx}) and precise (\figref{fig:precise}) visualization side by side.
This method reminds participants of what they have previously seen so that they can clear their memory from approximate results.
Lastly, we could overlay the approximate data as small ticks on the bar chart that shows the precise data (\figref{fig:diff}).

\subsubsection{Participants}

We recruited 34 participants from Amazon Mechanical Turk (AMT), a common tool to run perception studies for visualizations~\cite{heer2010crowdsourcing}.
Similar to other studies of human perception of visualizations, we restricted participation to people on AMT with at least 1,000 tasks completed and an approval rate above 95\%.
Participation was restricted to the United States to ensure familiarity with the airlines and states in the dataset.
Participants were compensated \$2.00 for completing the study.

\subsection{Measurements}
\subsubsection{Independent Variables}

In our study, each participant was asked to focus on a particular airline and state.
All questions were about the one airline and state directly or comparisons with the selected airline or state.
The state and airline to focus on were picked at random from the top three airlines and states and remained the same throughout the study.

%Probably make a figure for this too.
We evaluated three techniques for presenting the precise visualization (\secref{sec:vizes}).
Our control condition presented the precise data as both the approximate and precise data.
We further tested presenting the approximate and precise visualizations side-by-side, and a visualization highlighting the difference to approximate data on the precise visualization (\figref{fig:diff}).
The difference visualization aims to highlight the errors of the approximation as done in Pangloss \cite{moritz2017pangloss}.

We varied how much of the dataset was sampled for each approximate visualization.
We tested sample levels of 0.1\%, 0.2\%, 0.5\%, 1\%, and 10\%, with an additional control condition of 100\% (e.g., the approximate visualization was identical to the precise visualization).
Smaller samples tended to have greater difference from the precise visualizations.
For each approximate visualization, we recorded the difference from the precise visualization relative to the questions we asked.
For example, if a participant was asked ``How many flights were there on Alaska Airlines?'', we record the difference in number of flights between the approximate and precise visualization.

In this study, we randomized the order that participants saw the two visualizations in; some saw the visualization for airlines first, others saw the visualization of states first.
We used the same order for the approximate and precise visualizations.
We also randomized approximate sample, focus state, focus airline, and study condition for each participant.
As previously stated, the sample level of the approximate visualizations varied by visualization.

\subsubsection{Dependent Variables}

We asked three types of questions about each precise visualization, building on Brehmer and Munzner's task taxonomy \cite{brehmer2013multi}.
Participants were asked to recall (1) how many flights were there on or out of the focus state or airline, (2) the difference in flights between the focus state or airline and another state or airline in the visualization, and (3) which states or airlines were in the visualization.
We asked question (2) three times for each precise visualization with three different comparison airlines.
This repetition, as well as the repetition between state and airline, gave us repeated measures for each participant.

Each dependent variable has a "correct" answer (e.g., the data presented in the precise visualization).
We normalize our outcome (e.g., the answer participants gave) by the by the value of the "correct" answer, e.g.,

\[\frac{correct\_answer - answer\_given}{correct\_answer}\]

We refer to this value as the \textbf{measured bias}.

The experimental conditions inform how much variation there was between the precise and the approximate visualizations.
For each question we ask the participants, we compute the difference between the "correct" answer (e.g., the data presented in the precise visualization) and the "approximate" answer (e.g., the data presented in the precise visualization).
Again, we normalize this value by the "correct" answer, e.g.,

\[\frac{correct\_answer - approximate\_answer}{correct\_answer}\]

We refer to this value as the \textbf{expected bias}.
We treat the expected bias as an independent variable in our analysis.
We hypothesized that the greater the expected bias, the greater the measured bias would be.
Meaning, if an approximate visualization differed substantially from the precise visualization, we expected participant's answers to be close to correct (e.g., a measured bias of near 0), but skew in the direction of the approximate visualization.
We anticipated that the measured bias would be less than the expected bias, as participants had seen the updated, correct visualization.

For question (3), the answers were calculated via a Jaccard Difference between the airlines or states in the visualization and the airlines or states in the participant's answer.

\subsection{Participants}

We recruited 40 participants for the study.
One did not complete the study and 5 participants had to be removed because they said they cheated during the study.
They said they wrote down the data in the visualizations, despite the instructions asking them them not to.
We used the data from the remaining 34 participants for our analysis.

The average participant was 33.7 years old.
26 participants self-identified as male and 8 as female.
All participants reported that they had no technical problems with the study procedures.
Only 11\% of our participants said they are not with all of these visualization types: bar charts, scatter plots, and heatmaps.

\section{Results}\label{sec:results}

\subsection{Descriptive Statistics}\label{sec:desc}

In this study we wanted to understand whether analysts have a bias in how they recall precise data.
Before we analyze the bias between what participants remembered from the approximate and precise visualizations, we look at the relative errors that participants made when they answered questions about the visualizations.
\tabref{table_errors} summarized the results.
We expect the errors for the precise data to be larger than the approximate one because it is the first visualization they look at.
However, that is not true for type 1 questions because of a single outlier where a participant types in a value that was an order of magnitude off.
We rerun the statistics with outliers removed (sigma clipping with values outside of $4 \times sigma$ removed).
Now the error for the precise visualization is larger (non-paired t-test because missing data does not allow us to run a paired test any more, p=0.0002).
For questions of type 2, the errors and variances were much larger and we don't see significantly larger errors in the precise visualizations (paired t-test with mean as value per participant, p=0.032).

\begin{table}[!t]
% increase table row spacing, adjust to taste
\renewcommand{\arraystretch}{1.3}
% if using array.sty, it might be a good idea to tweak the value of
% \extrarowheight as needed to properly center the text within the cells
\caption{The errors for different questions. The results in parentheses are with outliers removed (sigma clipping with $4 \time sigma$).}
\label{table_errors}
\centering
\begin{tabular}{|l||r|r|r|r|}
\hline
Question & Data        & Mean error & Standard deviation \\ \hline
\hline
type 1   & approx      & 0.117 (0.021)     & 0.535 (0.035)             \\ \hline
type 1   & precise     & 0.099 (0.087)     & 0.167 (0.130)            \\ \hline
type 2   & approx      & 0.803 (0.420)     & 1.649 (0.528)            \\ \hline
type 2   & precise     & 2.075 (0.469)     & 8.500 (0.453)            \\ \hline
\end{tabular}
\end{table}

\tabref{table_bias} summarizes the expected and true biases.
For question type 1, both the measured and expected biases are relatively small (e.g., about 10\% of the correct values).
For question type 2, the measured biases averaged over 100\% of the correct values, with an even larger standard deviation.
This means that participant answers to question type 2 were off by orders of magnitude.

\begin{table}[!t]
% increase table row spacing, adjust to taste
\renewcommand{\arraystretch}{1.3}
% if using array.sty, it might be a good idea to tweak the value of
% \extrarowheight as needed to properly center the text within the cells
\caption{The mean and standard deviation of the measured and expected bias.}
\label{table_bias}
\centering
\begin{tabular}{|l||r|r|r|r|}
\hline
Question & Bias Type  & Mean  & Standard deviation \\ \hline
\hline
type 1   & measured  &  0.010 & 0.195  \\ \hline
type 1   & expected  & -0.011 & 0.215  \\ \hline
type 2   & measured  & -1.542 & 5.244  \\ \hline
type 2   & expected  & -0.145 & 0.691  \\ \hline
\end{tabular}
\end{table}

Participants completed the study (excluding instructions) in around 7:30 minutes (with a standard deviation of 3.27 minutes).
As expected, most participants spent more time on the questions than the visualizations and overall the most time on the questions about the precise data.

We asked participants to report the confidence in their responses.
\figref{fig:confidence} shows the distribution of relative frequencies grouped by approximate and precise.

\begin{figure}[!t]
  \centering
  \includegraphics[width=\columnwidth]{confidence}
  \caption{Relative counts of confidence in answers about the approximate and precise data. 1 is the lowest confidence, 7 the highest.}
  \label{fig:confidence}
\end{figure}

\subsection{Statistical Analysis}

We analyzed our data through linear regression using the \texttt{lm} package in R, analyzing the results with a factorial ANOVA (\texttt{anova} in R).
We hypothesized that multiple factors would impact our outcome metrics, so we used multiple regression in our analyses.
We treated each participant as a random effect.

As expected, visualization order, focus state and focus airline had no effect on our results for any of the tests we conducted.
For the comparison questions, the comparison state or airline also did not have an effect on our results.
None of these independent variables pertain to our research questions, so we discard the terms from our analysis and do not report on them further.

The regression tables of our three types of questions are in Tables~\ref{table_q1}, \ref{table_q2}, and \ref{table_q3}.
The raw data for our three types of questions are in Figures~\ref{figure_q1}, \ref{figure_q2}, and \ref{figure_q3}.
We now answer our research questions in turn.

\begin{table}[!t]
% increase table row spacing, adjust to taste
\renewcommand{\arraystretch}{1.3}
% if using array.sty, it might be a good idea to tweak the value of
% \extrarowheight as needed to properly center the text within the cells
\caption{Regression table for question type 1, how many flights were there on an airline or state. Baseline is no difference between the approximate and precise visualization, showing only the precise data in the precise visualization.}
\label{table_q1}
\centering
\begin{tabular}{|l||r|r|r|r@{}l|}
\hline
Study parameter & Est. & SE & t & p & \\
\hline
\hline
Intercept & -0.033 & 0.068 & -0.482 & 0.6313 &  \\
\hline
Approx$\neq$Precise & 0.214 & 0.060 & 0.360 & 0.7200 & * \\
\hline
Expected Bias & 0.074 & 0.254 & 0.290 & 0.7729 &  \\
\hline
VisType\textsubscript{Diff} & 0.014 & 0.055 & 0.255 & 0.7999 &  \\
\hline
VisType\textsubscript{Both} & 0.074 & 0.058 & 1.284 & 0.2041 &  \\
\hline
Expected Bias*VisType\textsubscript{Diff} & 0.072 & 0.286 & 0.252 & 0.8016 &  \\
\hline
Expected Bias*VisType\textsubscript{Both} & 0.791 & 0.317 & 2.492 & 0.0154 & * \\
\hline
\end{tabular}
\end{table}

\begin{table}[!t]
% increase table row spacing, adjust to taste
\renewcommand{\arraystretch}{1.3}
% if using array.sty, it might be a good idea to tweak the value of
% \extrarowheight as needed to properly center the text within the cells
\caption{Regression table for question type 2, comparing how many flights to another airline or state. Baseline is the same as question type 1.}
\label{table_q2}
\centering
\begin{tabular}{|l||r|r|r|r@{}l|}
\hline
Study parameter & Est. & SE & t & p & \\
\hline
\hline
Intercept & -2.312 & 1.066 & -2.168 & 0.0314 & * \\
\hline
Approx$\neq$Precise & 1.745 & 0.918 & 1.902 & 0.0586 & . \\
\hline
Expected Bias & 4.165 & 0.746 & 5.584 & $<$0.0001 & *** \\
\hline
VisType\textsubscript{Diff} & -0.230 & 0.850 & -0.271 & 0.7867 &  \\
\hline
VisType\textsubscript{Both} & 0.576 & 0.921 & 0.063 & 0.9502 &  \\
\hline
Expected Bias*VisType\textsubscript{Diff} & -1.474 & 0.994 & -1.482 & 0.1398 &  \\
\hline
Expected Bias*VisType\textsubscript{Both} & -2.247 & 1.808 & 1.243 & 0.2154 &  \\
\hline
\end{tabular}
\end{table}

\begin{table}[!t]
% increase table row spacing, adjust to taste
\renewcommand{\arraystretch}{1.3}
% if using array.sty, it might be a good idea to tweak the value of
% \extrarowheight as needed to properly center the text within the cells
\caption{Regression table for question type 3, which airlines or states were in the dataset. Baseline is the same as question type 1.}
\label{table_q3}
\centering
\begin{tabular}{|l||r|r|r|r@{}l|}
\hline
Study parameter & Est. & SE & t & p & \\
\hline
\hline
Intercept & 0.217 & 0.176 & 1.234 & 0.2220 &  \\
\hline
Approx$\neq$Precise & 0.009 & 0.033 & 0.279 & 0.7815 & \\
\hline
Expected Bias & 0.732 & 0.202 & 3.629 & 0.0006 & *** \\
\hline
VisType\textsubscript{Diff} & 0.233 & 0.208 & 1.119 & 0.268 &  \\
\hline
VisType\textsubscript{Both} & -0.108 & 0.202 & -0.533 & 0.5961 &  \\
\hline
Expected Bias*VisType\textsubscript{Diff} & -0.278 & 0.236 & -1.175 & 0.2447 & \\
\hline
Expected Bias*VisType\textsubscript{Both} & 0.122 & 0.232 & 0.526 & 0.6008 &  \\
\hline
\end{tabular}
\end{table}

\begin{figure}[!t]
  \centering
  \includegraphics[width=\columnwidth]{how_many_precise.pdf}
  \caption{Participant data for question type 1 plotted by measured and expected bias.}
  \label{figure_q1}
\end{figure}

\begin{figure}[!t]
  \centering
  \includegraphics[width=\columnwidth]{compare_precise.pdf}
  \caption{Participant data for question type 2 plotted by measured and expected bias.}
  \label{figure_q2}
\end{figure}

\begin{figure}[!t]
  \centering
  \includegraphics[width=\columnwidth]{jaccard_precise_jitter.pdf}
  \caption{Participant data for question type 3 plotted by measured and expected bias.}
  \label{figure_q3}
\end{figure}

\subsubsection{RQ1: Does incorrect approximate data bias responses about correct data?}

Our results suggest that viewing incorrect approximate data biases participant responses about correct data for certain types of questions.
For all three question types, the expected bias had a significant effect on participant response ($F_{1,61}=10.767, p=0.002$, $F_{1, 197}=58.399, p<0.0001$ for type 2, $F_{1, 61}=78.448, p<0.0001$ for type 3).

The effect for question type 1 was smaller than the other question types (95\% CI -0.435-0.582).
This aligns with the finding that the error for this type of question was lower overall.
For question types 2 and 3, the sign of the impact of the expected bias was in the same direction of the bias ($95\% CI 2.594-5.637, 0.329-1.135$ for type 2 and 3, respectively).
This suggests that people tend to be biased in the same direction as the data they see in the approximate visualization.
However, for question type 2, the measured bias was \textit{larger} than the expected bias.

This means that people are reporting answers closer to the approximate visualization than to the precise visualization.
One interpretation of this is that there is a significant bias.
However, we suspect participants remembered very little about the precise data.
Instead we believe this result is heavily influenced by outliers in participant responses.

\subsubsection{RQ2: Does visualization type impact the bias?}

Our results fail to reject the hypothesis that visualization type impacts the bias.
Condition did not have a significant main effect for any question type ($F_{2,61}=0.888, p=0.417$, $F_{2,197}=0.026, p=0.975$, and $F_{2,61}=0.205, p=0.815$ for types 1, 2, and 3 respectively).

We believe the individual variance between participants exceeded any differences between visualizations.
Our results suggest there may be interaction effects between visualization type and the impact of expected bias.
However, the variation in participant responses in general made us hesitant to probe deeper into these interaction effects.

\section{Discussion}\label{sec:discussion}

We are a bit unsure why we see the bias we do.
On one hand, we predicted we would see a bias in the data.
However, we expected the magnitude of the bias would be less than the expected error (e.g., people would be biased only a little by the imprecise visualization).
Instead, we seem to be seeing that people are just not remembering the data they saw.

In \secref{sec:desc}, we saw that the errors for type 2 questions were much larger than type 1.
This makes sense as we asked participants to focus on a particular state and airline and type 1 questions were specifically about that one item.
Still, even then errors are very large which could be because participants struggled with the comparison questions.
Remember, we asked participants to say how much more flights there were on the focus airline and state and they had to type a negative value if the airline or state we compared to had more flights than the focus airline or state.
A more detailed analysis is needed to tease apart why participants made such large errors.

Because errors in the responses from participants were so large, we have little statistical power to tease apart how highlighting differences may reduce biases to answer RQ2.
With more statistical power we might also be able to understand the role of how certain participants were in their responses.
For each question we asked participants to rate their confidence in their answer.
For an analyst it would be fatal to have a large bias and high confidence in their recollection.
However, if either the bias is low or the confidence is low, the analysts is likely to not make the wrong decision when it really matters.
In our data we did not see significant support for the hypothesis that high confidence correlates with lower biases.

\subsection{Threats to Validity}

We had no way of ensuring that participants did not take screenshots or notes in order to answer later questions about the precise visualizations.
Although we asked participants to not take notes during the study, a few self-reported that they did.
We removed these participants from the data but we don't know whether any other participant took notes in secret.
Another problematic factor is the large error for type 2 questions (see \tabref{table_errors}).
This could be because participants misunderstood the task or because they ignored the instructions.

We used linear regression in all of our statistical analyses.
We made this choice because our outcome is numeric rather than categorical.
Additionally, we had no reason to anticipate our data would fit a different distribution.
However, viewing Figures~\ref{figure_q1}, \ref{figure_q2}, and \ref{figure_q3} shows that although much of the data appears distributed around 0 measured bias, there are many participant responses far away from this value.
These responses do not appear to necessarily be outliers, meaning that many people were answering questions similarly wrong.

Linear regression follows a sum-of-squares approach, and is therefore heavily influenced by values far away from the mean.
Perhaps a different analysis technique is better suited to our data.
We believe our data has systematic problems in how participants are responding to the questions which are unlikely to be corrected by choosing a different statistical test.
However, future studies in this domain should ensure that outliers are being treated in a sensible way.

\section{Conclusion}

The questions we try to answer in this paper in a more general sense are what negative effects seeing a less precise model before a more precise one has on how well people recollect the more precise data.
We focused on the specific case where a user only sees one approximate model before the precise one but the findings should generalize to the progressive case because there is nothing inherently different about the precise model.
The problem of seeing imprecise models before precise ones is relevant in a number of domains from journalists who may show poll results before the votes, a machine learning expert who builds models, to an economist who presents the forecast of economical development.
In this paper, we focused on data exploration in approximate query processing systems with the goal of finding actionable insights that inform how we design such systems.

In this study, we found that exposure to approximate data prior to seeing the precise data biases the observations people make.
This has important implications for the design of applications that present approximate results as well as for journalists who report for example polls before votes.
In this paper we tried to reduce the bias by making people aware of the approximate result they have previously seen but we did not see a significant reduction of the bias.

In future work, we should focus of usable methods to reduce the bias that can be used by approximate visualization tools.
Also, we should also try to understand the relationship between confidence and bias to avoid that analysts make wrong decisions with high confidence.
Future work should also investigate how the findings from this paper generalize to other applications.

All material for this study, results, and the analysis are available at \url{https://github.com/domoritz/bias-study}.

% use section* for acknowledgment
\ifCLASSOPTIONcompsoc
  % The Computer Society usually uses the plural form
  \section*{Acknowledgments}
\else
  % regular IEEE prefers the singular form
  \section*{Acknowledgment}
\fi


The authors would like to thank Michael D. Ernst for his feedback on the experimental design and writing.
They would also like to thank Jessica Hullman and Danyel Fisher for their feedback on the study design and research questions.

% Can use something like this to put references on a page
% by themselves when using endfloat and the captionsoff option.
\ifCLASSOPTIONcaptionsoff
  \newpage
\fi



% trigger a \newpage just before the given reference
% number - used to balance the columns on the last page
% adjust value as needed - may need to be readjusted if
% the document is modified later
%\IEEEtriggeratref{8}
% The "triggered" command can be changed if desired:
%\IEEEtriggercmd{\enlargethispage{-5in}}

% references section

% can use a bibliography generated by BibTeX as a .bbl file
% BibTeX documentation can be easily obtained at:
% http://mirror.ctan.org/biblio/bibtex/contrib/doc/
% The IEEEtran BibTeX style support page is at:
% http://www.michaelshell.org/tex/ieeetran/bibtex/
\bibliographystyle{IEEEtran}
% argument is your BibTeX string definitions and bibliography database(s)
\bibliography{IEEEabrv,paper}
%
% <OR> manually copy in the resultant .bbl file
% set second argument of \begin to the number of references
% (used to reserve space for the reference number labels box)
%\begin{thebibliography}{1}
%\end{thebibliography}

% that's all folks
\end{document}


